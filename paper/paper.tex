\documentclass{sig-alternate}

\begin{document}

\title{State Of The Art Crowdsourcing Applications
\titlenote{(Produces the permission block, and
copyright information). For use with
SIG-ALTERNATE.CLS. Supported by ACM.}}
\subtitle{Advanced Internet Computing - Wintersemester 2013/14
\titlenote{A full version of this paper is available as
\textit{Author's Guide to Preparing ACM SIG Proceedings Using
\LaTeX$2_\epsilon$\ and BibTeX} at
\texttt{www.acm.org/eaddress.htm}}}

\numberofauthors{3}
\author{
\alignauthor
Simon Zuend\titlenote{He is a mastermind.}\\
       \affaddr{1025990}\\
       \email{e1025990@student\\.tuwien.ac.at}\\
\alignauthor
Martin Keiblinger\titlenote{He did tons of work.}\\
       \affaddr{0825118}\\
       \email{e0825118@student\\.tuwien.ac.at}
\alignauthor
Klaus Nigsch\titlenote{This author is the
one who did all the really hard work.}\\
       \affaddr{1025991}\\
       \email{e1025991@studen\\t.tuwien.ac.at}
}

\maketitle
\begin{abstract}

With Crowdsourcing transforming to the web, a new field rises. \cite{cswww}
Still, the prime examples of the Crowdsourcing (CS) paradigm are Wikipedia, 
Linux, Yahoo! Answers and Mechanical Turk-based systems.

Crowdsourcing Systems are on the rise, yet defining Crowdsourcing systems turns out to be
surprisingly tricky \cite{cswww}.

This article will give some insight on what is state of the art on Crowdsourcing Systems and
what we can expect of them in the future. 

\end{abstract}

\terms{Terms}

Crowdsourcing - Instead of using lots of computational power to solve complex tasks, let a multitude of humans solve these (for humans often easy) problems.


\section{Introduction}

\section{State of the Art}

To analyse the state of the art of Crowdsourcing, a good start is looking at some existing Crowdsourcing systems.
There is a wide variety of them, also aiming in totally different directions.

An interesting example is http://www.freelancer.com/contest/ .
If you want something implemented, request your idea as a contest. You also set a reward / a price for freelancers (basically developers) to complete this task.
And here exactly the idea of crowdsourcing applies: Human workers will solve the task for you, in different quality of course. You can then choose
from the solutions and award the best entry, provide feedback to the developers and finally starting a handover process to legally own that entry.

This already shows the possibilities of Crowdsourcing. It is by no means limited to small tasks or a small mass of human workers.
Everyone can register as freelancer and participate in the contest, and the task also does not have to be trivial for humans, but can also be complex.

\section{Future}

What can we expect from Crowdsourcing platforms?

\section{The {\secit Body} of The Paper}

%
% The following two commands are all you need in the
% initial runs of your .tex file to
% produce the bibliography for the citations in your paper.
\bibliographystyle{abbrv}
\bibliography{sigproc}  % sigproc.bib is the name of the Bibliography in this case
% You must have a proper ".bib" file


\end{document}
